%%% Exemplo de utilização da classe ITA
%%%
%%%   por        Fábio Fagundes Silveira   -  ffs [at] ita [dot] br
%%%              Benedito C. O. Maciel     -  bcmaciel [at] ita [dot] br
%%%              Giovani Volnei Meinertz   -  giovani [at] ita [dot] br
%%%    	         Hudson Alberto Bode       -  bode [at] ita [dot]br
%%%    	         P. I. Braga de Queiroz    -  pi [at] ita [dot] br
%%%    	         Jorge A. B. Gripp         -  gripp [at] ita [dot] br
%%%    	         Juliano Monte-Mor         -  jamontemor [at] yahoo [dot] com [dot] br
%%%    	         Tarcisio A. B. Gripp      -  tarcisio.gripp [at] gmail [dot] com
%%%    	         
%%%
%%%  IMPORTANTE: O texto contido neste exemplo nao significa absolutamente nada.  :-)
%%%              O intuito aqui eh demonstrar os comandos criados na classe e suas
%%%              respectivas utilizacoes.
%%%
%%%  Tese.tex  2015-04-08
%%%  $HeadURL: http://www.apgita.org.br/apgita/teses-e-latex.php $
%%%
%%% ITALUS
%%% Instituto Tecnológico de Aeronáutica --- ITA, Sao Jose dos Campos, Brasil
%%%                   http://groups.yahoo.com/group/italus/
%%% Discussion list: italus {at} yahoogroups.com
%%%
%++++++++++++++++++++++++++++++++++++++++++++++++++++++++++++++++++++++++++++++
% Parametros da classe ITA para inserir em \documentclass[?]{?}
%   tg       = Trabalho de Graduacao
%   tgfem    = Para Engenheiras
%   msc      = Dissertacao de Mestrado
%   mscfem   = Para Mestras
%   dsc      = Tese de Doutorado
%   dscfem   = Para Doutoras
%   quali    = Exame de Qualificacao
%   qualifem = Exame de Qualificacao para Doutoras
%   dv       = 'Draft Version'     --> imprime 'Versao Preliminar + data no rodape
%   eng      = para teses em inglês
%++++++++++++++++++++++++++++++++++++++++++++++++++++++++++++++++++++++++++++++
%se fosse em inglês: \documentclass[dsc, eng]{ita}
%para ``draft version'': \documentclass[dsc, dv]{ita} ou \documentclass[dsc, eng, dv]{ita}

\documentclass[tg, eng]{ita}    % ITA.cls based on standard book.cls 
% Quando alterar a classe, por exemplo de [msc] para [msc, eng]) rode mais uma vez o botão BUILD OUTPUT caso haja erro
\usepackage{ae}
\usepackage{graphicx}
\usepackage{epsfig}
\usepackage{amsmath}
\usepackage{amssymb} 
\usepackage{subfig}
\usepackage{multirow}
\usepackage{float}

%++++++++++++++++++++++++++++++++++++++++++++++++++++++++++++++++++++++++++++++
% Espaçamento padrão de todo o documento
%++++++++++++++++++++++++++++++++++++++++++++++++++++++++++++++++++++++++++++++
\onehalfspacing

%singlespacing Para um espaçamento simples
%onehalfspacing Para um espaçamento de 1,5
%doublespacing Para um espaçamento duplo

%++++++++++++++++++++++++++++++++++++++++++++++++++++++++++++++++++++++++++++++
% Identificacoes (se o trabalho for em inglês, insira os dados em inglês)
% Para entradas abreviadas de Professora (Profa.) em português escreva: Prof$^\textnormal{a}$.
%++++++++++++++++++++++++++++++++++++++++++++++++++++++++++++++++++++++++++++++
\course{Computer Engineering} % Programa de PG ou Curso de Graduação
\dept{Computer Engineering} % Divisão Acadêmica no ITA

% Autor do trabalho: Nome Sobrenome
\authorgender{masc}                     %sexo: masc ou fem
\author{Márcio Valença}{Ramos}
\itaauthoraddress{Rua H8B, 239}{12228-461}{São José dos Campos--SP}

% Titulo da Tese/Dissertação
\title{Relate Kanji: A Statistical Analysis Of The Japanese Language And Consequences For Teaching Methods}

% Orientador
\advisorgender{masc}                    % masc ou fem
\advisor{Prof. Dr.}{Carlos Henrique Costa Ribeiro}{ITA}

% Coorientador (Caso não haja coorientador, colocar ambas as variáveis \coadvisorgender e \coadvisor comentadas, com um % na frente)
%\coadvisorgender{fem}									% masc ou fem
%\coadvisor{Prof$^\textnormal{a}$.~Dr$^\textnormal{a}$.}{Doralice Serra}{OVNI}

% Pró-reitor da Pós-graduação
%\bossgender{masc}												% masc ou fem
%\boss{Prof.~Dr.}{NAO USADO}

%Coordenador do curso no caso de TG
\bosscoursegender{fem}									% masc ou fem
\bosscourse{Prof.Dr.}{Cecília de Azevedo Castro César}

% Palavras-Chaves informadas pela Biblioteca -> utilizada na CIP
\kwcip{Kanji}
\kwcip{Language}
\kwcip{Linguistics}
\kwcip{Statistics}
\kwcip{Pedagogy}
\kwcip{Andragogy}

% membros da banca examinadora

\examiner{Prof. Dr.}{Carlos Henrique Costa Ribeiro}{}{ITA}
\examiner{Prof. Dr.}{Clóvis Torres Fernandes}{}{ITA}

% Data da defesa (mês em maiúsculo, se trabalho em inglês, e minúsculo se trabalho em português) 
\date{21}{DECEMBER}{2016}

% Número CDU - (somente para TG)
\cdu{TBD}

% Glossario
\makeglossary
\frontmatter

\begin{document}
% Folha de Rosto e Capa para o caso do TG
\maketitle

% Dedicatoria: Nao esqueca essa secao  ... :-)
\begin{itadedication}
I dedicate this thesis to... TBW
\end{itadedication}

% Agradecimentos
\begin{itathanks}
\input{Cap0/agradecimentos}
\end{itathanks}

% Epígrafe
\thispagestyle{empty}
\ifhyperref\pdfbookmark[0]{\nameepigraphe}{epigrafe}\fi
\begin{flushright}
\begin{spacing}{1}
\mbox{}\vfill
{\sffamily\itshape
``Understanding is a kind of ecstasy.''\\}
--- \textsc{Carl Sagan}
\end{spacing}
\end{flushright}

% Resumo
%\begin{abstract}
%\input{Cap0/resumo}
%\end{abstract}

% Abstract
\begin{englishabstract}
\input{Cap0/abstract}
\end{englishabstract}

% Lista de figuras
\listoffigures %opcional

% Lista de tabelas
\listoftables %opcional

% Lista de abreviaturas
\listofabbreviations
\input{Cap0/listaabreviaturas} %opcional

% Lista de simbolos
\listofsymbols
\input{Cap0/listasimbolos} %opcional

% Sumario
\tableofcontents

\mainmatter
% Os capitulos comecam aqui

\chapter{Introduction}
\section{Kanji Complexity Overview}
\section{Japanese Language History}
\section{Present approaches}
\subsection{SKIP and SPAN}
\subsection{Learning Books}
\subsubsection{A guide to Remembering Japanese Kanji}
\subsubsection{Pictographic Kanji}
\subsection{Anki}
\subsection{Wani-Kani}
\section{Advantages of Interactive Media}

\chapter{Linguistic Frequency Distribution}
\section{Estimating Frequency Distribution}
\subsection{Caveats}
\section{The Zipf Distribution}
\section{General Word Distribution}
\section{Jouyou Kanji Distribution}
\section{Approximate Pareto Rule Implications}

\chapter{Dealing With Kanji Readings}
\section{Motivation}
\section{Parsing Japanese Characters}
\subsection{Parallels to Compilers Theory}
\subsection{Brute Force Approach}
\section{Enhancing Results Through Reading Types Labels}
\section{General Reading Distributions}
\section{Proportion of Regular to Irregular Kanji Use}
\section{Comparing official readings with measured frequencies}

\chapter{Working With Graphs}
\section{Building Graphs}
\subsection{Morphological Graph}
\subsection{Co-occurrences Graph}
\section{Random Walk Through the Graph}
\subsection{Theoretical Framework}
\subsection{Parallels With a Real Student}
\subsection{Caveats}
\subsection{Application}
\subsection{Modifying the Random Teleport Parameter}
\section{Teaching Methods For Native Japanese Students}

\chapter{Applying Statistical Knowledge To Teaching Methods}
\section{Applications to Problems Related to Leveling Students}
\section{Modifying Spaced Repetition Models}
\section{Modelling User Interaction}
\section{Dangers of Gamefication Over-application}
\section{Multidimensional Flashcards}
\section{Reading Exercises}
\section{An Improved Kanji Search Tool Using Tries}

\chapter{Conclusion}
\input{Cap6/cap6}

% REFERENCIAS BIBLIOGRAFICAS
\renewcommand\bibname{\itareferencesnamebabel} %renomear título do capítulo referências
\bibliography{Referencias/referencias}

% Apendices
\appendix
\chapter{Placeholder for Appendix} %opcional
This annex serves to clarify technical terms and concepts from the field of linguistics that are relevant to the development of this academic project.

\section{Japanese as a Moraic System}\label{mora}
A Mora is a unit in phonology that can be defined as "something of which a long syllable consists of two and a short syllable consists of one"\cite{mccawley1968}. As per this definition, the Japanese is said to be moraic, since two of its base scripts (Hiragana and Katakana) translate strongly the concept of mora to its characters. 

For example, we can refer to the word \={O}saka, which could also be spelled as Oosaka (with a double "o") or in Hiragana \jap{おおさか}. This word would be be broken in three syllables: \={o}/sa/ka, but as denoted by the macron over the "o" letter, this is a long vowel. As can be seen in the Hiragana representation of this word, we have four graphemes, exactly the same as the number of morae. Though this is not always the case in Japanese, it is so in the great majority of cases, representing the importance of morae in the understanding of Japanese.

% Anexos
\annex
\chapter{Placeholder for Annex} %opcional
\input{AneA/anexoA}

% Glossario
%\itaglossary
%\printglossary

% Folha de Registro do Documento
% Valores dos campos do formulario
\FRDitadata{21 de Dezembro de 2016}
\FRDitadocnro{DCTA/ITA/TC-018/2016} %(o número de registro você solicita a biblioteca)
\FRDitaorgaointerno{Instituto Tecnológico de Aeronáutica -- Computer Engineering Division -- ITA}
%Exemplo no caso de pós-graduação: Instituto Tecnol{\'o}gico de Aeron{\'a}utica -- ITA

\FRDitapalavrasautor{Kanji; Language; Linguistics; Statistics; Pedagogy; Andragogy}
\FRDitapalavrasresult{Kanji; Language; Statistics}
%Exemplo no caso de graduação (TG):
\FRDitapalavraapresentacao{Trabalho de Graduação, ITA, São José dos Campos, 2016. \NumPenultimaPagina\ páginas.}
%Exemplo no caso de pós-graduação (msc, dsc):
%\FRDitapalavraapresentacao{ITA, São José dos Campos. Curso de Mestrado. Programa de Pós-Graduação em Engenharia Aeronáutica e Mecânica. Área de Sistemas Aeroespaciais e Mecatrônica. Orientador: Prof.~Dr. Adalberto Santos Dupont. Coorientadora: Prof$^\textnormal{a}$.~Dr$^\textnormal{a}$. Doralice Serra. Defesa em 05/03/2015. Publicada em 25/03/2015.}
\FRDitaresumo{\input{Cap0/abstract}}
%  Primeiro Parametro: Nacional ou Internacional -- N/I
%  Segundo parametro: Ostensivo, Reservado, Confidencial ou Secreto -- O/R/C/S
\FRDitaOpcoes{N}{O}
% Cria o formulario
\itaFRD

\end{document}
% Fim do Documento. O massacre acabou!!! :-)
