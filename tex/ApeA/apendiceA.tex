This annex serves to clarify technical terms and concepts from the field of linguistics that are relevant to the development of this academic project.

\section{Japanese as a Moraic System}\label{mora}
A Mora is a unit in phonology that can be defined as "something of which a long syllable consists of two and a short syllable consists of one"\cite{mccawley1968}. As per this definition, the Japanese is said to be moraic, since two of its base scripts (Hiragana and Katakana) translate strongly the concept of mora to its characters. 

For example, we can refer to the word \={O}saka, which could also be spelled as Oosaka (with a double "o") or in Hiragana \jap{おおさか}. This word would be be broken in three syllables: \={o}/sa/ka, but as denoted by the macron over the "o" letter, this is a long vowel. As can be seen in the Hiragana representation of this word, we have four graphemes, exactly the same as the number of morae. Though this is not always the case in Japanese, it is so in the great majority of cases, representing the importance of morae in the understanding of Japanese.